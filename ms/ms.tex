% Options for packages loaded elsewhere
\PassOptionsToPackage{unicode}{hyperref}
\PassOptionsToPackage{hyphens}{url}
%
\documentclass[
]{article}
\title{The French domestic Pass Sanitaire did not solve vaccination
inequities: a nationwide longitudinal study on 64.5 million individuals}
\author{F. Débarre, E. Lecoeur, L. Guimier, M. Jauffret-Roustide, A.-S.
Jannot}
\date{}

\usepackage{amsmath,amssymb}
\usepackage{lmodern}
\usepackage{iftex}
\ifPDFTeX
  \usepackage[T1]{fontenc}
  \usepackage[utf8]{inputenc}
  \usepackage{textcomp} % provide euro and other symbols
\else % if luatex or xetex
  \usepackage{unicode-math}
  \defaultfontfeatures{Scale=MatchLowercase}
  \defaultfontfeatures[\rmfamily]{Ligatures=TeX,Scale=1}
\fi
% Use upquote if available, for straight quotes in verbatim environments
\IfFileExists{upquote.sty}{\usepackage{upquote}}{}
\IfFileExists{microtype.sty}{% use microtype if available
  \usepackage[]{microtype}
  \UseMicrotypeSet[protrusion]{basicmath} % disable protrusion for tt fonts
}{}
\makeatletter
\@ifundefined{KOMAClassName}{% if non-KOMA class
  \IfFileExists{parskip.sty}{%
    \usepackage{parskip}
  }{% else
    \setlength{\parindent}{0pt}
    \setlength{\parskip}{6pt plus 2pt minus 1pt}}
}{% if KOMA class
  \KOMAoptions{parskip=half}}
\makeatother
\usepackage{xcolor}
\IfFileExists{xurl.sty}{\usepackage{xurl}}{} % add URL line breaks if available
\IfFileExists{bookmark.sty}{\usepackage{bookmark}}{\usepackage{hyperref}}
\hypersetup{
  pdftitle={The French domestic Pass Sanitaire did not solve vaccination inequities: a nationwide longitudinal study on 64.5 million individuals},
  pdfauthor={F. Débarre, E. Lecoeur, L. Guimier, M. Jauffret-Roustide, A.-S. Jannot},
  hidelinks,
  pdfcreator={LaTeX via pandoc}}
\urlstyle{same} % disable monospaced font for URLs
\usepackage[margin=1in]{geometry}
\usepackage{graphicx}
\makeatletter
\def\maxwidth{\ifdim\Gin@nat@width>\linewidth\linewidth\else\Gin@nat@width\fi}
\def\maxheight{\ifdim\Gin@nat@height>\textheight\textheight\else\Gin@nat@height\fi}
\makeatother
% Scale images if necessary, so that they will not overflow the page
% margins by default, and it is still possible to overwrite the defaults
% using explicit options in \includegraphics[width, height, ...]{}
\setkeys{Gin}{width=\maxwidth,height=\maxheight,keepaspectratio}
% Set default figure placement to htbp
\makeatletter
\def\fps@figure{htbp}
\makeatother
\setlength{\emergencystretch}{3em} % prevent overfull lines
\providecommand{\tightlist}{%
  \setlength{\itemsep}{0pt}\setlength{\parskip}{0pt}}
\setcounter{secnumdepth}{-\maxdimen} % remove section numbering
\newlength{\cslhangindent}
\setlength{\cslhangindent}{1.5em}
\newlength{\csllabelwidth}
\setlength{\csllabelwidth}{3em}
\newlength{\cslentryspacingunit} % times entry-spacing
\setlength{\cslentryspacingunit}{\parskip}
\newenvironment{CSLReferences}[2] % #1 hanging-ident, #2 entry spacing
 {% don't indent paragraphs
  \setlength{\parindent}{0pt}
  % turn on hanging indent if param 1 is 1
  \ifodd #1
  \let\oldpar\par
  \def\par{\hangindent=\cslhangindent\oldpar}
  \fi
  % set entry spacing
  \setlength{\parskip}{#2\cslentryspacingunit}
 }%
 {}
\usepackage{calc}
\newcommand{\CSLBlock}[1]{#1\hfill\break}
\newcommand{\CSLLeftMargin}[1]{\parbox[t]{\csllabelwidth}{#1}}
\newcommand{\CSLRightInline}[1]{\parbox[t]{\linewidth - \csllabelwidth}{#1}\break}
\newcommand{\CSLIndent}[1]{\hspace{\cslhangindent}#1}
\usepackage{setspace}
\usepackage{lineno} % line numbering
\linenumbers

% Line spacing
\usepackage{setspace}
%\doublespacing

\usepackage{geometry}
\geometry{top=2.5cm, bottom=3cm, left=4cm, right=4cm}

% Fonts of the main text
\usepackage{fourier}
% Helvetica for sans serif fonts
\renewcommand{\sfdefault}{phv}

\usepackage[font={small,sf, doublespacing}, labelfont=bf]{caption}
\captionsetup{format=hang, font={it}, labelfont={bf}} % Format of captions
\captionsetup[table]{position=bottom}   %% or below
\ifLuaTeX
  \usepackage{selnolig}  % disable illegal ligatures
\fi

\begin{document}
\maketitle

\hypertarget{affiliations}{%
\subsection{Affiliations}\label{affiliations}}

FD: Institute of Ecology and Environmental Sciences of Paris
(IEES-Paris, UMR 7618), CNRS, Sorbonne Université, UPEC, IRD, INRAE,
75252 Paris, France\\
EL: Clinical research unit, Hôpital Européen Georges Pompidou, APHP,
Paris, France\\
LG: French Institute of Geopolitics, Paris, France\\
MJR: Centre d'Étude des Mouvements Sociaux (Inserm U1276/CNRS
UMR8044/EHESS), Paris, France.\\
British Columbia Center For Substance Use (BCCSU), Vancouver, Canada\\
Bady Center on Law and Social Policy, Buffalo, NY, USA\\
AJR: Medical Informatics, biostatistics and public health unit, Hôpital
Européen Georges Pompidou, APHP, Paris, France\\
Université de Paris, Paris, France\\
HeKA, Centre de Recherche des Cordeliers, Inserm, INRIA, Paris, France

\emph{Corresponding author:} Anne-Sophie Jannot, Hôpital Européen
Georges Pompidou, 20 rue Leblanc, 75015 Paris.
\texttt{annesophie.jannot@aphp.fr}

\hypertarget{keywords}{%
\subsubsection{Keywords:}\label{keywords}}

vaccination, covid-19, data mining, socio-economic factors, health
inequities.

\hypertarget{acknowledgements}{%
\subsection{Acknowledgements}\label{acknowledgements}}

We thank the producers of public datasets, in particular David Levy at
INSEE and Antoine Rachas at Assurance Maladie.

\hypertarget{funding}{%
\subsection{Funding}\label{funding}}

EL received funding to match socio-economic data with medical data from
AP-HP Centre Université de Paris.

\hypertarget{contributions}{%
\subsection{Contributions}\label{contributions}}

ASJ and FD designed the study with inputs from all authors. EL extracted
socio-economic and political orientation data at district scale and
computed indicators. ASJ, EL and FD had full access to aggregated data
used for this study and take responsibility for the integrity of the
data. EL and FD did the analyses and takes responsibility for the
accuracy of the data analysis. FD drafted the paper with the help of
ASJ, MR. All authors critically revised the manuscript for important
intellectual content and gave final approval for the version to be
published.

\hypertarget{conflict-of-interest-statement}{%
\subsection{Conflict of interest
statement}\label{conflict-of-interest-statement}}

No conflict of interest to disclose

\hypertarget{data-sources}{%
\subsection{Data sources}\label{data-sources}}

\begin{itemize}
\tightlist
\item
  Assurance Maladie:
  \url{https://datavaccin-covid.ameli.fr/explore/dataset/donnees-devaccination-par-epci/}
  \url{https://datavaccin-covid.ameli.fr/explore/dataset/donnees-de-vaccination-parcommune/information/}
\item
  INSEE: \url{https://www.insee.fr/fr/statistiques/5359146\#consulter}
\end{itemize}

\hypertarget{abstract}{%
\section{Abstract}\label{abstract}}

Context: To encourage Covid-19 vaccination, France introduced during the
Summer 2021 a ``Sanitary Pass,'' which morphed into a ``Vaccine Passe''
in early 2022. While the Sanity Pass led to an increase in Covid-19
vaccination rates, spatial heterogeneities in vaccination rates
remained. To identify potential determinants of these heterogeneities
and evaluate the French Sanitary and Vaccine Pass' efficacies in
reducing them, we used a data-driven approach on exhaustive nationwide
data, gathering 141 socio-economic, political and geographic indicators.

Methods: We considered the association between being a district above
the median value of the first-dose vaccination rates and being above the
median value of each indicator at different time points: just before the
sanitary pass announcement (week 2021-W27), just before the sanitary
pass came into force (week 2021-W31) and one month after (week
2021-W35), and the equivalent dates for the vaccine pass (weeks
2021-W49, 2022-W03, 2022-W07). We then considered the change over time
of vaccination rates according to deciles of the three of the most
associated indicators.

Results: The indicators most associated with vaccination rates were the
share of local income coming from unemployment benefits, the proportion
of overcrowded households, the proportion of immigrants in the district,
and vote for an ``anti-establishment'' candidate at the 2017
Presidential election. Vaccination rate also were also contrasted along
a North-West -- South East axis, with lower vaccination coverage in the
South-East of France.

Conclusion: Our analysis reveals that, both before and after the
introduction of the French sanitary and vaccination passes, factors with
the largest impact are related to poverty, immigration, and trust in the
government.

\hypertarget{introduction}{%
\section{Introduction}\label{introduction}}

The rapid development of effective COVID-19 vaccines brought the hope of
a rapid return to normalcy, but heterogeneous vaccination rates, both
among countries because of inequitable distributions of doses (Usher
2021) and within countries (Caspi et al. 2021; Murthy et al. 2021),
jeopardize epidemic control.

Hesitancy and hostility toward vaccination have been comparatively high
in France in recent decades (European Commission. Directorate General
for Health and Food Safety. 2018). Modern vaccine hesitancy in France
started with claims of a link between the hepatitis B vaccine and
multiple sclerosis (J. K. Ward et al. 2019); it strongly increased
following the 2009-2010 vaccination campaign against pandemic flu, the
contested management of which in France was a tipping point that led to
higher vaccine hesitancy and hostility (Guimier 2021; J. K. Ward et al.
2019). The trend was confirmed with the COVID-19 pandemic (Lindholt et
al. 2021; Spire, Bajos, and Silberzan 2021): just before Covid-19
vaccines became available, intentions to get vaccinated were
comparatively very low in France compared to other countries (44\% of
the respondents in (Wouters et al. 2021) in the Fall 2020; about 40\% of
respondent in (Santé Publique France 2021) in December 2020). Acceptance
of the COVID-19 vaccine however gradually grew during 2021 (Santé
Publique France 2021; J. Ward 2021).

Spatial heterogeneties in vaccination rates have already been documented
in France for previous vaccines. Vaccination coverage for the Hepatitis
B vaccine and for the Measles-Mumps-Rubella vaccine has been lower in
the South of France, and especially in the South-East of the country
(Guimier 2021). Distance to the central political power in Paris, as
well as a sense of belonging to a local community with a strong cultural
identity, have been put forward as potential explanations for this
geographic gradient in vaccination rates (Guimier 2021).

Attitudes toward vaccination are also known to be influenced by social
and territorial inequalities. Surveys conducted in 2020 in France showed
that respondents with lower education (Schwarzinger et al. 2021; Coulaud
et al. 2022), lower income levels or less trust in authorities (Spire,
Bajos, and Silberzan 2021; Lindholt et al. 2021) were more likely to be
hostile to COVID-19 vaccines.

By mid-July 2021, France was facing an epidemic wave due to the Delta
variant. To speed up vaccination, President Macron announced on 12 July
2021 the implementation of a domestic ``sanitary pass'' (le passe
sanitaire), which came fully into force on 9 August 2021. Presenting as
a QR code, a long-term sanitary pass was obtained after full vaccination
(two doses, or only one dose in the case of a documented previous
Covid-19 infection), and a short-term version could be obtained with a
negative Covid-19 test. The ``sanitary pass'' was required in most
cultural venues, for both indoor and outdoor dining and in health
structures. This announcement led to an unprecedented demand for
vaccination (Oliu-Barton et al. 2022), which was considered
internationally as a potential model to follow. Vaccination rates
climbed from about 64\% of the population over 20 years old by 11 July
2021 (52\% of all ages) to 82 on 5 September 2021 (69\% of all ages).
Because it targeted pay-for social activities, however, the ``sanitary
pass'' was feared to have a limited impact on vaccination inequities. By
mid-December 2021, at the height of the winter Delta wave, and while the
Omicron wave was looming, the French Prime Minister announced that the
Sanitary Pass would become a Vaccine Pass, i.e.~that a negative Covid-19
test would not provide a temporary QR code any longer for adults --
making vaccination implicitly mandatory in France. The Vaccine Pass came
into force on 24 January 2022.

This study aims to obtain further insights on the socio-economic,
political and geographic factor associated with vaccination rates, and
to evaluate the effect of the French domestic sanitary pass, by using
nation-wide, exhaustive datasets.

\hypertarget{methods}{%
\section{Methods}\label{methods}}

\hypertarget{data}{%
\subsection{Data}\label{data}}

The French state health insurance service (Assurance Maladie) provides
public datasets of vaccination rates in France. These datasets are based
on aggregated individual data on beneficiaries of the national health
insurance service who received health care in the past year. These
exhaustive datasets are updated weekly, and are provided at the district
scale nationally (EPCI: \emph{Établissement public de coopération
intercommunale}, an administrative level gathering multiple towns or
cities) and at the suburban scale for the Paris, Lyon, and Marseille
metropolitan areas. For this study, we focused on mainland France,
because vaccination rates are much lower in oversea localities, and
because determinants of vaccination rates are likely to differ in
oversea localities compared to mainland ones. Our dataset included 1555
districts (1228 EPCIs and 327 districts at the suburban scale in Paris,
Lyon, Marseille).

The vaccination dataset for mainland France encompasses about 64.5
million individuals (median district size 22310 inhabitants,
interquartile range 11012--43038). The vaccination data are available by
age class: 00--19, 20--39, 40--54, 55--64, 65--74, 75 and over.
Population sizes for each locality and each age class are also provided.

We paired these vaccination data with three other datasets gathering
socio-economic, political orientation and geographic variables.

Socio-economic data are provided by the French national statistics
institute (INSEE), and are available at the same administrative levels
as the vaccination data. We selected the most recent dataset available
(year 2018). The different variables available in the dataset are
classified by INSEE according to 8 categories (Activity, Education,
Employment, Family, Housing, Immigration, Income, Population).\\
Latitude, longitude and surface data were extracted from open geographic
datasets. We calculated from them four additional geographic indicators:
distance to Paris, relative position along a South-East--North-West
gradient, relative position along a South-West--North-East gradient, and
local population density.\\
Political orientation data consisted of the results of the 2017
Presidential election in France, which we aggregated to reconstitute the
same administrative levels as the vaccination dataset. This political
dataset contains the proportions of votes for each of the 11 candidates
of the first round, 2 candidates of the second round (Macron and Le
Pen), and proportion of abstention at each round.

These three datasets comprised 312 indicators. We then removed those
indicators with over 5\% missing data, or with over 0.9 correlation with
other indicators of the dataset, which left us with 141 indicators: 123
socio-economic indicators (Activity: n = 10; Education: n = 16;
Employment: n = 25; Family: n = 20; Housing: n = 30; Immigration: n = 1;
Income: n = 13; Population: n = 8); 6 geographic indicators; 12
political indicators.

\hypertarget{analysis}{%
\subsection{Analysis}\label{analysis}}

Vaccination was accessible to all adults in France after 27 May 2021. It
opened to teenagers (12-17 year olds) on 15 June 2021, and to children
(5-11 year olds) on 22 December 2021. Because of this differential
accessibility of vaccines, and because vaccine passport rules also
differed for non-adults, we excluded the 00-19 age class from our
analysis, and focused on vaccination rates among 20+ year-old
individuals (hereafter ``adults'').

For each indicator in our dataset, at each of the four chosen dates
(weeks 2021-W27, 2021-W31, 2021-W35, 2021-W49, 2022-W03, 2022-W07), we
considered the association between living in a district above the median
of a that indicator and individual first-dose vaccination rates among
adults. Odds ratios (OR) were computed from the output of a logistic
regression. To be able to compare predictors irrespective of the
direction of the effect, we considered the maximum of \texttt{OR,\ 1/OR}
(hereafter \(\overline{OR}\)). Note that vaccination data are at the
individual level, and indicator data at the district level. The analysis
is done at the individual level, with indicators characterizing the
geographic districts in which individuals live.\\
For each date, we determined a significance threshold by computing odds
ratios on 1000 random permutations of a predictor, and identifying the
value of the 99\% percentile odd ratios (\(\overline{OR}\)) of these
permuted data.

For representative indicators among the most statistically significantly
associated ones, we estimated standardized vaccination rates among
adults over time, for each decile of each indicator (treated as a
factor). These estimations were obtained from a logistic model taking
age class into account; adult vaccination rates were standarsized using
an age distribution matching that of mainland France.

All analysis code is available at XXX; analyses were done in R version
4.0.4 (2021-02-15).

\hypertarget{results}{%
\section{Results}\label{results}}

We investigated the associations between each of the 141 indicators
characterizing districts of residence, and the fact of having received
at least one Covid-19 vaccine dose, on the whole population of mainland
France. Two indicators were among the top five most associated one at
all time points (see Figure @ref(fig:figManhattan)): the share of local
income coming from unemployment benefits (\texttt{Unemployment\_Benef};
strongest association on 2022-01-23, \(OR = 0.716\)) and vote for the
``anti-establishment'' political party represented by the candidate
Asselineau (\texttt{Asselineau}; \(OR = 0.712\) on 2022-02-20). The
three other most associated indicators did not change in the later dates
that we considered, and were the proportion of immigrants in the
district (\texttt{Immigrant}; \(OR = 0.713\) on 2022-02-20), the
district's relative position along a North-West--South-East gradient
(\texttt{NO-SE}; \(OR = 0.745\) on 2021-12-12) and the proportion of
overcrowded households (\texttt{Overcrowding\_rate}; \(OR = 0.738\) on
2022-01-23).

\begin{figure}
\centering
\includegraphics{ms_files/figure-latex/figManhattan-1.pdf}
\caption{Manhattan plots of the Odds ratios for each of the indicator of
our dataset, by date. Left column: around the Sanitary Pass
implementation; right column: around the Vaccine Pass implementation.
The top odds ratios are labelled at each time point; the symbol next to
the name indicates the direction of the effect. The gray rectangle
corresponds to the 99\% percentile of odds ratios in the permuted data;
points falling in the rectangle are considered as non-significant.}
\end{figure}

Our odds ratio calculations were based on a crude version of each
indicator, which were dichotomized into values above or below the median
of each indicator. To better visualize the effects (or lack thereof) of
the sanitary and vaccine passes on vaccination rates over time, we
computed age-adjusted vaccination rates over time, by decile of three of
the most associated indicators, treated as factors (see Figure
@ref(fig:figOverTime)). The Sanitary Pass, implemented in the Summer
2021, led to an overall increase in vaccination rates; on the other
hand, the Vaccine Pass, implemented in the end of 2021, did not affect
the evolution of vaccination rates. Heterogeneities in vaccination rates
persisted after both types of pass; vaccination rates gradually decrease
by decile of each indicator, confirming the association of these
indicators with vaccination rates without threshold effect. Of note, for
the Unemployment and Asselineau vote indicators, the difference between
the 9th and 10th deciles appears to be much larger than between the
other consecutive deciles.

\begin{figure}
\centering
\includegraphics{ms_files/figure-latex/figOverTime-1.pdf}
\caption{Age-adjusted vaccination rates among adults, over time, by
decile of each indicator (presented by a color gradient). The vertical
lines indicate the dates of announcements and implementations of the
sanitary and vaccine passes.}
\end{figure}

Finally, historically under-vaccinated areas in France stand out as
being less vaccinated against Covid-19, in particular the South-East
region (see Figure @ref(fig:figMap)).

\begin{figure}
\centering
\includegraphics{ms_files/figure-latex/figMap-1.pdf}
\caption{Adult vaccination rates by district of mainland France}
\end{figure}

\hypertarget{discussion}{%
\section{Discussion}\label{discussion}}

Our results, based on exhaustive national datasets, indicate that the
French sanitary pass, and the later vaccine pass, did not solve Covid-19
vaccination heterogeneities, but instead crystallized them. Indicators
most associated with vaccination rates were associated to poverty,
immigration, anti-establishment vote (or abstention), and a North-West
-- South-East contrast. For instance, the odds for an adult to still be
unvaccinated by the end of February 2021 are about 1.4 times higher when
living in the districts with higher than median value share of income
coming from unemployment benefits, than when living in the districts
with lower than median value.

The indicators associated to vaccination rates can be interpreted in the
light of the dimensions of vaccine hesitancy (J. K. Ward et al. 2022). A
first reason for vaccine hesitancy is complacency: not fully perceiving
the benefit of vaccination or the risks of severe disease. While in this
case a sanitary or vaccine pass may convince complacent individuals to
get vaccinated, it is less efficient if the associated constrain is low.
As the French domestic pass was associated to pay-for activities
(restaurants, tourism), its persuading effect could be limited among
poorer populations. This may explain the association of lower
vaccination rate with poverty in the data that we analyzed: vaccination
rates decrease as the share of local income coming from unemployment
benefits (\texttt{Unemployment\_Benef}) or the proportion of overcrowded
households (\texttt{Overcrowding\_rate})) increase.\\
A second reason for vaccine hesitancy is confidence, i.e.~trust in the
vaccine, in the health care system, and more generally in the government
(J. K. Ward et al. 2022; Lindholt et al. 2021). A survey conducted in
July 2021 in France confirmed that trust in the government and trust in
scientists were associated to higher odds to be vaccinated (Bajos et al.
2022). Votes for Mr Asselineau -- which represented a minority of cast
votes in 2017 in France (less than \(1\%\) overall) -- can be
interpreted as mistrust in the government (or more generally, against
the establishment): This candidate for instance proposed that France
exits the European Union, leave the Euro zone and reinstall the Franc
currency; he was a proponent of hydroxychloroquin and ivermectin during
the Covid-19 pandemic, and publicly expressed doubts about the safety of
available Covid-19 vaccines. The association of higher proportion of
votes for Mr Asselineau with lower vaccination rates can be interpreted
as revealing a lack of confidence for the government. Noteworthily,
among political indicators, the second strongest association is with
abstention rates (higher abstention rates being associated to lower
vaccination rates), again signaling higher distrust for institutions (J.
K. Ward et al. 2020). Likewise, the lower vaccinations rates in the
South-East of France can be interpreted as mistrust of the central
government in Paris.\\
Finally, a third reason for vaccine hesitancy is convenience, that is,
the availability and accessibility of the vaccines (J. K. Ward et al.
2022). During the first half of 2021, vaccination rate in France was
mostly constrained by dose availability. Vaccination slots were to be
booked online, and there was no general system for sending individual
invitations to get vaccinated. It is therefore still possible that, in
spite of some local outreach efforts, vaccine accessibility remained an
issue, which may explain at least part of the association of lower
vaccination rates with poverty. These accessibility issues may also
explain the association we find between lower vaccination rates and
living in a district with a high proportion of immigrants, which may for
instance reveal language barriers. These associations of lower
vaccination rates with more poverty and with higher proportions of
immigrants in the district of residence are compatible with the results
of a survey conducted in July 2021 in France (Bajos et al. 2022) on
close to 81000 participants, which indicated that unvaccinated
respondents were more likely to have lower income and more likely to
belong to racialised minorities than vaccinated respondents

Relative position of the district of residency along a
North-West--South-East gradient is also associated with vaccination
probability, the South-East being less vaccinated. This geographic
feature, already documented for other kinds of vaccination (Guimier
2021), have been shown to be the consequence of multiple determinants
with a common consequence: a local climate of mistrust for the central
Parisian power. Politically, anti-system votes (from the right as well
as from the left) are traditionally concentrated in the South-East of
France. Medically, General Practitioners (GPs) based in the South-East,
and to a lesser extent those in the South-West, have been shown to tend
to have a more negative opinion of vaccination than their colleagues
practicing in the northern part of France (Gautier, Jestin, and Beck
2013). This greater skepticism influences GP practices and attitudes,
resulting in a lesser degree of compliance with vaccination schedules
than GPs in the northern half of France (Collange et al. 2015). Physical
distance to the central government and institutions, based in Paris,
coupled with a sense of belonging to a local community with a strong
cultural identity, as is the case for example in the Marseille
metropolis or in the Cévennes, play a role in indifference or mistrust
towards institutions perceived as distant authorities (Guimier 2021).
Finally, in and around the Marseille metropolis, the image of a
rebellious territory was reinforced since the first months of the
epidemic in France through the hypermediatized Pr Didier Raoult. Based
in Marseille, he was a promoter of a controversial treatment against
Covid-19 based on hydroxychloroquine and azithromycin (Schultz et al.
2022), and later held ambiguous positions regarding Covid-19
vaccination. He has become a local icon, thanks to his anti-system
positions, and against the hostility of most of the medical world
towards his work. All in all, around the city of Marseille, and more
broadly in South-Eastern France, the climate of suspicion against
Parisian institutions, which had long been rooted in the area, hardened
during the Covid-19 crisis, and was associated with distrust of Covid-19
vaccines.

The design of our study offers several advantages. First, we used a
data-driven approach, i.e.~we did not focus on indicators that we
\emph{a priori} thought to be associated with vaccination rate. The
indicators that we identified as the most associated with vaccination
rates were not biased towards our previous knowledge or surveys about
vaccine hesitancy. Secondly, the data that we used are real-world data
on effective vaccination, and not vaccination intentions. Intentions to
be vaccinated and realized vaccination may not always match, especially
with the introduction of measures like the French Sanitary and Vaccine
Passes. For instance, according to a survey conducted in the Fall 2021,
the introduction of the sanitary pass led to an increase in the share of
individuals reporting being ``angry they had to be vaccinated'' (J. K.
Ward et al. 2022). From an immediate public health perspective, such as
the limitation of the number of severe cases, realized vaccination rates
are a more useful metric. Finally, the vaccination data we used are
based on records of the national health insurance service: they cover
64.5 million individuals living in mainland France, and vaccination
rates are not self-reported, which strongly limits reporting bias.

Still, the design of our study also presents limitations. While our
vaccination data are at the individual level, the socio-economic,
political and geographic indicators are at the district level, and must
therefore be interpreted as such: for instance, we cannot not show that
receiving unemployment benefits is associated with lower vaccination
probability, but we find an association with lower vaccination
probability and the fact of living in a district where a large share of
income comes from unemployment benefits. In addition, although the
different indicators are analysed independently in our study, their
combinations may affect vaccination rates. For instance, the effect of
mistrust in the government on vaccination refusal was shown to be even
stronger among individuals from lower social classes than from higher
social classes (Bajos et al. 2022). Finally, our data do not inform
directly on the reasons for non-vaccination -- e.g., whether it is
hesitancy, refusal, or accessibility issues, which is why our approach
is complementary to qualitative surveys.

To conclude, by emphasizing a differentiated use of COVID-19 vaccination
according to a socio-economic gradient, our study confirms the strong
impact of social inequalities on COVID-19. Previous research found that
the most deprived areas have been disproportionately infected and
hospitalized during the pandemic (Jannot et al. 2021; Bajos et al.
2021). We further show that poorer districts are also the least
vaccinated and, hence, the most still at risk, despite the widely
celebrated domestic sanitary pass. There is an urgent need to define new
vaccination policies that truly address social inequities.

\hypertarget{references}{%
\section*{References}\label{references}}
\addcontentsline{toc}{section}{References}

\hypertarget{refs}{}
\begin{CSLReferences}{1}{0}
\leavevmode\vadjust pre{\hypertarget{ref-bajosWhenLockdownPolicies2021}{}}%
Bajos, Nathalie, Florence Jusot, Ariane Pailhé, Alexis Spire, Claude
Martin, Laurence Meyer, Nathalie Lydié, et al. 2021. {``When Lockdown
Policies Amplify Social Inequalities in {COVID-19} Infections: Evidence
from a Cross-Sectional Population-Based Survey in {France}.''} \emph{BMC
Public Health} 21 (1): 705.
\url{https://doi.org/10.1186/s12889-021-10521-5}.

\leavevmode\vadjust pre{\hypertarget{ref-bajosWhenMistrustGovernment2022}{}}%
Bajos, Nathalie, Alexis Spire, Lena Silberzan, Antoine Sireyjol,
Florence Jusot, Laurence Meyer, Jeanna-Eve Franck, and Josiane
Warszawski. 2022. {``When Mistrust in the Government and Scientists
Reinforce Social Inequalities in Vaccination Against {Covid-19}.''}
Preprint. {Public and Global Health}.
\url{https://doi.org/10.1101/2022.02.23.22271397}.

\leavevmode\vadjust pre{\hypertarget{ref-caspiSocioeconomicDisparitiesCOVID192021}{}}%
Caspi, Gil, Avshalom Dayan, Yael Eshal, Sigal Liverant-Taub, Gilad Twig,
Uri Shalit, Yair Lewis, Avi Shina, and Oren Caspi. 2021.
{``Socioeconomic Disparities and {COVID-19} Vaccination Acceptance: A
Nationwide Ecologic Study.''} \emph{Clinical Microbiology and
Infection}, June, S1198743X21002779.
\url{https://doi.org/10.1016/j.cmi.2021.05.030}.

\leavevmode\vadjust pre{\hypertarget{ref-Collange2015}{}}%
Collange, Fanny, Lisa Fressard, Pierre Verger, Fanny Josancy, Rémy
Sebbah, Arnaud Gautier, Christine Jestin, et al. 2015. {``Vaccinations :
Attitudes Et Pratiques Des Médecins Généralistes.''} 910. {Direction de
la recherche, des études, de l'évaluation et des statistiques}.

\leavevmode\vadjust pre{\hypertarget{ref-coulaudCOVID19VaccineIntention2022}{}}%
Coulaud, Pierre-julien, Aidan Ablona, Naseeb Bolduc, Danya Fast, Karine
Bertrand, Jeremy K. Ward, Devon Greyson, Marie Jauffret-Roustide, and
Rod Knight. 2022. {``{COVID-19} Vaccine Intention Among Young Adults:
{Comparative} Results from a Cross-Sectional Study in {Canada} and
{France}.''} \emph{Vaccine} 40 (16): 2442--56.
\url{https://doi.org/10.1016/j.vaccine.2022.02.085}.

\leavevmode\vadjust pre{\hypertarget{ref-europeancommission.directorategeneralforhealthandfoodsafety.StateVaccineConfidence2018}{}}%
European Commission. Directorate General for Health and Food Safety.
2018. \emph{State of Vaccine Confidence in the {EU} 2018.} {LU}:
{Publications Office}.

\leavevmode\vadjust pre{\hypertarget{ref-Gautier2013a}{}}%
Gautier, Arnaud, Christine Jestin, and François Beck. 2013.
{``Vaccination : Baisse de l'adhésion de La Population Et Rôle Clé Des
Professionnels de Santé.''} \emph{La Santé En Action} 423 (2013-03):
50--53.

\leavevmode\vadjust pre{\hypertarget{ref-guimierResistancesFrancaisesAux2021}{}}%
Guimier, Lucie. 2021. {``Les Résistances Françaises Aux Vaccinations :
Continuité Et Ruptures à La Lumière de La Pandémie de {Covid-19}:''}
\emph{Hérodote} N\textdegree{} 183 (4): 227--50.
\url{https://doi.org/10.3917/her.183.0227}.

\leavevmode\vadjust pre{\hypertarget{ref-jannotLowincomeNeighbourhoodWas2021}{}}%
Jannot, Anne-Sophie, Hector Countouris, Alexis Van Straaten, Anita
Burgun, Sandrine Katsahian, and Bastien Rance. 2021. {``Low-Income
Neighbourhood Was a Key Determinant of Severe {COVID-19} Incidence
During the First Wave of the Epidemic in {Paris}.''} \emph{Journal of
Epidemiology and Community Health}, June, jech-2020-216068.
\url{https://doi.org/10.1136/jech-2020-216068}.

\leavevmode\vadjust pre{\hypertarget{ref-lindholtPublicAcceptanceCOVID192021}{}}%
Lindholt, Marie Fly, Frederik Jørgensen, Alexander Bor, and Michael Bang
Petersen. 2021. {``Public Acceptance of {COVID-19} Vaccines:
Cross-National Evidence on Levels and Individual-Level Predictors Using
Observational Data.''} \emph{BMJ Open} 11 (6): e048172.
\url{https://doi.org/10.1136/bmjopen-2020-048172}.

\leavevmode\vadjust pre{\hypertarget{ref-murthyDisparitiesCOVID19Vaccination2021}{}}%
Murthy, Bhavini Patel, Natalie Sterrett, Daniel Weller, Elizabeth Zell,
Laura Reynolds, Robin L. Toblin, Neil Murthy, et al. 2021.
{``Disparities in {COVID-19 Vaccination Coverage Between Urban} and
{Rural Counties} \textemdash{} {United States}, {December} 14,
2020\textendash{{April}} 10, 2021.''} \emph{MMWR. Morbidity and
Mortality Weekly Report} 70 (20): 759--64.
\url{https://doi.org/10.15585/mmwr.mm7020e3}.

\leavevmode\vadjust pre{\hypertarget{ref-oliu-bartonEffectCOVIDCertificates2022}{}}%
Oliu-Barton, Miquel, Bary SR Pradel, Nicolas Woloszko, Lionel
Guetta-Jeanrenaud, Philippe Aghion, Patrick Artus, Arnaud Fontanet,
Philippe Martin, and Guntram B Wolff. 2022. {``The Effect of {COVID}
Certificates on Vaccine Uptake, Health Outcomes, and the Economy.''}
Preprint. {In Review}.
\url{https://doi.org/10.21203/rs.3.rs-1242919/v2}.

\leavevmode\vadjust pre{\hypertarget{ref-santepubliquefranceCoviPrevEnquetePour2021}{}}%
Santé Publique France. 2021. {``{CoviPrev} : Une Enquête Pour Suivre
l'évolution Des Comportements Et de La Santé Mentale Pendant l'épidémie
de {COVID-19}.''}

\leavevmode\vadjust pre{\hypertarget{ref-schultzDoesPublicKnow2022}{}}%
Schultz, Émilien, Laëtitia Atlani-Duault, Patrick Peretti-Watel, and
Jeremy K. Ward. 2022. {``Does the Public Know When a Scientific
Controversy Is over? {Public} Perceptions of Hydroxychloroquine in
{France} Between {April} 2020 and {June} 2021.''} \emph{Therapies},
January, S0040595722000105.
\url{https://doi.org/10.1016/j.therap.2022.01.008}.

\leavevmode\vadjust pre{\hypertarget{ref-schwarzingerCOVID19VaccineHesitancy2021}{}}%
Schwarzinger, Michaël, Verity Watson, Pierre Arwidson, François Alla,
and Stéphane Luchini. 2021. {``{COVID-19} Vaccine Hesitancy in a
Representative Working-Age Population in {France}: A Survey Experiment
Based on Vaccine Characteristics.''} \emph{The Lancet Public Health} 6
(4): e210--21. \url{https://doi.org/10.1016/S2468-2667(21)00012-8}.

\leavevmode\vadjust pre{\hypertarget{ref-spireSocialInequalitiesHostility2021}{}}%
Spire, Alexis, Nathalie Bajos, and Léna Silberzan. 2021. {``Social
Inequalities in Hostility Toward Vaccination Against {Covid-19}.''}
Preprint. {Public and Global Health}.
\url{https://doi.org/10.1101/2021.06.07.21258461}.

\leavevmode\vadjust pre{\hypertarget{ref-usherBeautifulIdeaHow2021}{}}%
Usher, Ann Danaiya. 2021. {``A Beautiful Idea: How {COVAX} Has Fallen
Short.''} \emph{The Lancet} 397 (10292): 2322--25.
\url{https://doi.org/10.1016/S0140-6736(21)01367-2}.

\leavevmode\vadjust pre{\hypertarget{ref-wardPremiersResultatsEnquete2021}{}}%
Ward, Jeremy. 2021. {``{Premiers résultats de l'enquête SLAVACO Vague 1
et approfondissement de l'analyse de l'enquête COVIREIVAC - les français
et la vaccination}.''}

\leavevmode\vadjust pre{\hypertarget{ref-wardFrenchPublicAttitudes2020}{}}%
Ward, Jeremy K., Caroline Alleaume, Patrick Peretti-Watel, Patrick
Peretti-Watel, Valérie Seror, Sébastien Cortaredona, Odile Launay, et
al. 2020. {``The {French} Public's Attitudes to a Future {COVID-19}
Vaccine: {The} Politicization of a Public Health Issue.''} \emph{Social
Science \& Medicine} 265 (November): 113414.
\url{https://doi.org/10.1016/j.socscimed.2020.113414}.

\leavevmode\vadjust pre{\hypertarget{ref-wardFrenchHealthPass2022}{}}%
Ward, Jeremy K., Fatima Gauna, Amandine Gagneux-Brunon, Elisabeth
Botelho-Nevers, Jean-Luc Cracowski, Charles Khouri, Odile Launay, Pierre
Verger, and Patrick Peretti-Watel. 2022. {``The {French} Health Pass
Holds Lessons for Mandatory {COVID-19} Vaccination.''} \emph{Nature
Medicine}, January. \url{https://doi.org/10.1038/s41591-021-01661-7}.

\leavevmode\vadjust pre{\hypertarget{ref-wardVaccineHesitancyCoercion2019}{}}%
Ward, Jeremy K., Patrick Peretti-Watel, Aurélie Bocquier, Valérie Seror,
and Pierre Verger. 2019. {``Vaccine Hesitancy and Coercion: All Eyes on
{France}.''} \emph{Nature Immunology} 20 (10): 1257--59.
\url{https://doi.org/10.1038/s41590-019-0488-9}.

\leavevmode\vadjust pre{\hypertarget{ref-woutersChallengesEnsuringGlobal2021}{}}%
Wouters, Olivier J, Kenneth C Shadlen, Maximilian Salcher-Konrad, Andrew
J Pollard, Heidi J Larson, Yot Teerawattananon, and Mark Jit. 2021.
{``Challenges in Ensuring Global Access to {COVID-19} Vaccines:
Production, Affordability, Allocation, and Deployment.''} \emph{The
Lancet} 397 (10278): 1023--34.
\url{https://doi.org/10.1016/S0140-6736(21)00306-8}.

\end{CSLReferences}

\end{document}
